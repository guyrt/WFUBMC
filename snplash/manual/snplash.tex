\section{SNPlash}
\label{sec:snplash}

\subsection{Description}
SNPLASH (\underline{S}ingle \underline{N}ucleotide \underline{P}olymorphism
\underline{l}inkage \underline{a}ssociation \underline{s}oftware
\underline{h}ub) is a toolkit for genome-wide association studies.  It provides
facilities for calculating allele frequencies, Hardy-Weinberg statistics,
linkage disequilibrium, association (logistic and linear regression) and more.

SNPLASH contains the following modules which are described in more detail in the
sections following:

\begin{description}

  \item[ADTree]  \hfill \\
    This is what ADTree does

  \item[BADTree] \hfill \\
    (\underline{B}agged \underline{A}lternating \underline{D}ecision Tree)
    Detect sets of SNPs that associate with phenotype.

  \item[Dandelion] \hfill \\
    Infer haplotypes using Zaykin's method (Zaykin, 2002).

  \item[DPrime] \hfill \\
    Calculate the $r^2$ and $D'$ statistic for all combinations of SNPs.

  \item[InterTwoLog] \hfill \\
    Logistic test for association using a two-way interaction term.

  \item[QSNPGWA] \hfill \\
    \underline{Q}uantitative \underline{SNP} \underline{G}enome \underline{W}ide
    \underline{A}ssociation.  Linear regression association tests with a
    quantitative trait and optional covariates.

  \item[SNPGWA] \hfill \\
    Logistic regression with a binomial trait (case/control) and optional
    covariates. 

\end{description}

\subsection{Usage}
SNPLASH reads data in linkage format. Typing 'snplash' at the command line with
no options will print the program options. Usage common to all modules is:

\begin{verbatim}
  snplash -engine <engine> -geno <filename> -phen <filename> \
          -map <filename> -out <filename> [OPTIONS]

where,

  -engine   One of adtree, bagging, dandelion, dprime, intertwo-
            log, qsnpgwa or snpgwa (required)
  -geno     Genotype file in linkage format
  -bed      Genotype file in Plink binary format (v.0.991 or later)
  -phen     Phenotype file in linkage format (required)
  -map      Map file in linkage foramt (required)
  -out      Filename for output (required)
  -skipcols Comma-separated list of columns to skip in the geno-
            type file; this is processed before -beg and -end
  -beg      Column with the first SNP to include
  -end      Column with the last SNP
  -ign      Exclude individuals with missing data for any SNP
  -v        Control level of printing (not supported by all algo-
            rithms), select 1,2 or 3
\end{verbatim}                       
      
One of \verb|-geno| or \verb|-bed| is required.

\subsection{Output}

A header section is printed with information about the files read and options
used.  The variable labels are typically multi-line.  A listing of the output
variables is given for each module.

%% End snplash.tex \\
