\section{Dandelion}
\label{sec:dandy}
\subsection{Description}
%% Taken from the MS Word Document ``DANDELION User Guide, Version 3.0, February
%% 29, 2008'''

DANDELION is a program that reads a specific range of multiallelic or diallelic
marker data from an input file containing genetic data for case and control
individuals and calculates:

\begin{itemize}
\item Haplotype frequency maximum likelihood estimates for groups of cases,
  controls and combined cases and controls

\item A haplotype specific Z test statistic and p-value for each possible
  haplotype in the specified marker range

\item The odds ratio and 95\% confidence interval for each possible haplotype in
  the specified marker range

\item  Likelihood ratio statistic and p-value over the specified marker range

\item Multiallelic D Prime and diallelic R Squared statistics between each pair
  of markers in the specified marker range

\item An empirical p-value based on the
  likelihood ratio statistic (optional) 

\item All possible haplotype pairs that an individual could possess given the
  individual's genotype along with the probabilility that the individual
  possesses a specific haplotype pair (personal probability option)

\end{itemize}

The term "marker range" as it is used in this document refers to the contiguous
range of markers in the input file.  The marker range is specified at the
command line.  See ``Usage'', p.\pageref{sec:usage}.  DANDELION has a maximum
range of approximately 20 diallelic markers.  The suggested maximum marker range
is 15 diallelic markers.  If markers are not diallelic then smaller ranges are
suggested.

DANDELION implements the Expectation-Maximization (EM) Algorithm to calculate
haplotype frequency maximum likelihood estimates (Slatkin and Excoffier, 1996).
DANDELION performs the calculations described above using only haplotype
frequency maximum likelihood estimates with significant magnitude (values
greater than 0.0001).  DANDELION allows a maximum of two integers to represent
the alleles of a diallelic marker in the input file and only allows integers
between 1 and 30 to represent alleles.

\subsubsection{Haplotype Z Test Statistic and P-Value}
DANDELION calculates a Z statistic and p-value for each haplotype that has
significant probability.  The haplotype Z test is set up as two-proportion Z
test for two independent samples.  The two independent samples are groups of
cases and controls.  The Z statistic is calculated using haplotype frequency
maximum likelihood estimates as the proportions.

Let $n_1$ and $n_2$ be the numbers of individuals from case and control groups
respectively in the sample represented in the input file.  Similarly, let $p_1$
and $p_2$ be the respective haplotype frequency maximum likelihood estimates for
a specific haplotype from groups of cases and controls.  Then the Z statistic
and p-value for the haplotypes are calculated by

\begin{equation*}
  \hat{p} = \frac{n_1p_1 + n_2p_2}{n_1 + n_2}, \hat{q} = 1-\hat{p}
\end{equation*}

\begin{equation*}
  s_{p_1-p_2} = \sqrt{\hat{p}\hat{q}}\sqrt{\frac{n_1+n_2}{n_1n_2}}
\end{equation*}

The Z-statistic is 

\begin{equation*}
Z = \frac{p_1-p_2}{s_{p_1-p_2}}
\end{equation*}

\subsubsection{Haplotype Odds Ratio and 95\% Confidence Interval}

In addition to the Z statistic and p-value, DANDELION calculates a odds ratio
and 95\% confidence interval for each haplotype that has significant
probability.

DANDELION calculates the odds ratio by:

\begin{align*}
  n_{11} &= 2p_1n_1 \\
  n_{12} &= 2p_2n_2 \\
  n_{21} &= 2(1-p_1)n_1 \\
  n_{22} &= 2(1-p_2)n_2 \\
\end{align*}

The odds ratio is
\begin{equation*}
  OR = \sqrt{\frac{(n_{11}+0.5)(n_{22}+0.5)}{(n_{12}+0.5)(n_{21}+0.5)}}
\end{equation*}

The 95\% confidence interval is calculated by \{TODO: Is the following
correct?\}

\begin{equation*}
  LOR = \sqrt{\frac{1}{n_{11}+0.5}+\frac{1}{n_{12}+0.5}\frac{1}{n_{21}+0.5}\frac{1}{n_{22}+0.5}}
\end{equation*}

The lower and upper bounds of the 95\% confidence interval are

\begin{align*}
  LCI &= e^{ln(OR)-(1.96*LOR)} \\
  UCI &= e^{ln(OR)+(1.96*LOR)} \\
\end{align*}

\subsubsection{Likelihood Ratio Statistic}
\label{subsub:lrs}
DANDELION calculates a likelihood ratio statistic (LRS) (Green, Langefeld and
Lange, 2001) and LRS p-value for the specified marker range.  The LRS is a
measure of heterogeneity in the observed haplotype frequencies between cases and
controls.  DANDELION calculates the LRS using haplotype frequency maximum
likelihood estimates for groups of cases, controls and combined cases and
controls.  The LRS for each marker, two consecutive markers and three
consecutive markers is calculated as follows:

\begin{equation*} %% Space does not matter in equations!
  LRS = 2*log\frac{\displaystyle\prod_{i=1}^N Pr(g_i|\tilde{f}_{cases}) \prod_{j=N+1}^{2N} (g_j|\tilde{f}_{controls}) }
                  {\displaystyle\prod_{k=1}^{2N}Pr(g_k|\tilde{f}_{combined})}
\end{equation*}

\noindent{}where $g_i$ represents the multilocus genotypes in the $i^{th}$  individual.

LRS is distributed as a chi square variable with the degrees of freedom set to
the number of haplotypes that have significant autosomal haplotype frequencies
(greater than or equal to 0.0001) in either the case or control groups minus
one.


For example, suppose DANDELION calculates the LRS for three consecutive
diallelic markers.  In the process DANDELION employs the EM algorithm to
calculate haplotype frequency maximum likelihood estimates within the three
consecutive markers for groups of case individuals, control individuals and
combined case and control individuals.  Since each marker is diallelic there are
eight possible haplotypes that an individual could possess.  If DANDELION
determines five haplotypes in the case or control group have significant
probability (greater than or equal to 0.0001), then the LRS is distributed as a
chi square variable with 5 - 1 = 4 degrees of freedom.

DANDELION allows LRS based empirical p-value calculation.  See Sections
``Empirical P-Value Calculation'', p.\pageref{subsec:emp_p_val} and ``Empirical
P-Value Calculation from the Command Line'', p.\pageref{subsec:p_val_cmd_line} for
information on empirical p-value calculation.

\subsubsection{Linkage Disequilibrium Tests}

For each pair of markers, DANDELION calculates values for D Prime ($D'$) and R
Squared ($R^2$) linkage disequilibrium statistics (Devlin and Risch, 1995).
DANDELION calculates R Squared for diallelic markers only.  DANDELION calculates
the linkage disequilibrium statistics from haplotype frequency maximum
likelihood estimates calculated with the EM algorithm.  

Let A and B be two diallelic markers with alleles 1 and 2 and let the table
below show the haplotype frequencies as calculated by the EM Algorithm.  E.g.,
$f_{11}$is the frequency of haplotypes with allele 1 at marker 1 and allele 1 at marker
2.

\vspace{1em}
\begin{tabular}[!h]{ccccc}
  \hline
  & \multicolumn{4}{c}{MarkerA} \\
  \cline{2-5}
  \multirow{4}{*}{Marker B} & & Allele 1 & Allele 2 & \\
  & Allele 1 & $f_{11}$ & $f_{12}$ & $f_{1+}$ \\
  & Allele 2 & $f_{21}$ & $f_{22}$ & $f_{2+}$ \\
  &          & $f_{+1}$ & $f_{+2}$ & 1 \\
  \hline
\end{tabular}
\vspace{1em}

\noindent{}The linkage disequilibrium statistics are calculated as follows:

\begin{equation*}
  D = f_{11}f_{22} - f_{12}f_{21}
\end{equation*}

\begin{equation*}
D>0,~~~~    D' = \frac{D}{min(f_{1+}f_{+2}, f_{+1}f_{2+})}
\end{equation*}

\begin{equation*}
D<0,~~~~   D' = \frac{D}{min(f_{1+}f_{+1}, f_{+2}f_{2+})}
\end{equation*}

\begin{equation*}
  R^2 = \frac{D}{\sqrt{f_{1+}f_{2+}f_{+1}f_{+2}}}
\end{equation*}

Where $f_{1+}f_{2+}f_{+1}f_{+2}$ are marginal haplotype freqencies.

\subsubsection{Empirical P-Value Calculation}
\label{subsec:emp_p_val}

DANDELION has an option to calculate an LRS based empirical p-value using a
permutation process.  DANDELION calculates an LRS as described in Section
``Likelihood Ratio Statistic'', p.\pageref{subsub:lrs}, using unmodified sample
data prior to calculating an empirical p-value for that test.  This LRS is a chi
square value that necessary to calculate the empirical p-value.  This document
refers to it as the unpermuted LRS value ($\chi^2_u$).

DANDELION calculates an empirical p-value for a test by executing the following
three steps a specified number of times.  This document refers to the number of
times that DANDELION executes steps 1 through 3 as the number of permutations.

\begin{enumerate}
\item Permute the affection statuses (case or control) of the entire sample
  represented in the input file while preserving the total number of cases and
  total number of controls.
\item Recalculate the LRS value as described above p.\pageref{subsub:lrs} using
  the permuted sample.
\item Compare the LRS value calculated with permuted data with the unpermuted
  LRS value.  The LSR value calculated using permuted data is referred to as the
  permuted LSR value ($\chi^2_u$).
\end{enumerate}

DANDELION records the number of instances that the permuted LRS value exceeds
the unpermuted LRS value.  The permuted LRS value is considered to exceed the
unpermuted LRS value when the permuted LRS value is greater than the sum of the
unpermuted LRS value and a small epsilon.  Epsilon is set to 0.0001 in
DANDELION.  If a permuted LRS value is within epsilon of the unpermuted LRS
value, above or below, then DANDELION adds 0.5 to the number of instances the
permuted LRS value exceeds the unpermuted LRS value.  The empirical p-value is
the number of instances that the permuted LRS value exceeds the unpermuted LRS
value divided by the total number of permutations.  See the equation below.

\begin{equation*}
  EmpPVal = \frac{\displaystyle\sum_{i=1}^{NumPerms}\left\{ I\left[ \chi_{p_i}^2 > \chi_u^2 +\epsilon \right] + \frac{1}{2}I\left[ \chi_u^2 + \epsilon \geq \chi_{p_i}^2 > \chi_u^2 - \epsilon \right] \right\} }
                 {NumPerms}
\end{equation*}

\noindent{}Where $I\left[~\right]$ is an indicator function equal to 1 when the
statement in the bracket is true and $\epsilon=0.0001$ .  The total number of permutations is set
from the DANDELION command line.

\subsubsection{Personal Probability Calculation}
For each individual and given the individual's genotype, DANDELION calculates the
probability of possible haplotype pairs within the specified marker range that
the individual could possess.  The probability that an individual has a specific
haplotype pair is calculated using the EM algorithm generated haplotype
frequency maximum likelihood estimates with significant probability.

%% Section ``Input/Output''' omitted as obsolete with RTG implementation of
%% linkage-formated input.  DRM 30-Sep-2015

\subsection{Usage}
\label{sec:usage}
See ``SNPlash'', p.\pageref{sec:snplash}, for options common to all engines.

\begin{verbatim}
  snplash -engine dandelion -geno <filename> -pheno <filename> \
          -out <filename> -map <filename> --dandelion_pprob \
          --dandelion_window <integer>

where

  --dandelion_pprob   Create <outfile>.pprob with each indi-
                      vidual's probability of having each 
                      haplotype. (optional)
  --dandelion_window  Size of the set of snps in the file to
                      process (optional)
\end{verbatim}

%% ToDo: Fix cross-references
\subsubsection{Empirical P-Value Calculation from the Command Line}
\label{subsec:p_val_cmd_line}
As described in the section ``Empirical P-Value Calculation'',
p.\pageref{subsec:emp_p_val}, DANDELION has an option to calculate an LRS based
empirical p-value using a permutation process.

\{TODO: Check code to see if this still pertains.  It is not output in the help
as an option:\} 

DANDELION calculates empirical p-value only if the command line
contains the -per flag followed by the number of permutations.

For example, "DANDELION -geno genoFile -phen phenFile -out outFile -trait ds1
-per 100" tells DANDELION to calculate empirical p-values using 100
permutations.

Permutation testing requires random number generation.  DANDELION implements the
Wichman-Hill random number generator to generate random numbers (Wichman and
Hill, 1982).  The Wichman-Hill random number generator generates a specific
sequence of random numbers when configured with a specific sequence of three
integer seeds.  The default sequence is 1, 2 and 3.  A difference sequence may
designated from the command line with the -seed flag followed by a three integer
sequence (-seed integer1 integer2 integer3).

\subsection{Output}

DANDELION writes all output except personal probability results to a single
output file.  If DANDELION is executed with the option to calculate personal
probabilities, personal probability results are written to a separate file.

The output file contains the case, control and combined case and control haplotype frequencies, Z statistic and p-value, odds ratio and confidence interval for each haplotype in columnar format.

The output file also contains the likelihood ratio statistic and p-value, D
Prime and R Square statistics for each pair of markers.  DANDELION writes the
empirical p-value to the output file, if an empirical p-value is calculated.
See the example below.

\begin{verbatim}
*********************************
Map Markers: 
   Map Marker #1   : Marker01
   Map Marker #2   : Marker02
   Map Marker #3   : Marker03
   Map Marker #4   : Marker04
   Map Marker #5   : Marker05
   Map Marker #6   : Marker06
*********************************

                   <-EM Algorithm Est Haplotype Freqs-> <-Haplo. Z Stat & PVal->
       Haplotype       Cases     Controls    Combined   Z Statistic    P-Value   Odds Ratio      95% CI
----------------  ----------   ----------  ----------   -----------  ----------  ---------- ------------
4  1  2  3  3  2  0.28655010   0.32551301  0.30638950   1.49240592   0.13559277        0.83 (0.70, 0.99)
4  1  2  3  3  4  0.02968700   0.03953450  0.03413547   0.95149345   0.34135395        0.75 (0.48, 1.15)
4  1  2  3  1  2  0.04536407   0.03629012  0.03950821  -0.80875290   0.41865729        1.26 (0.85, 1.87)
   .
   .
   .
Haplotype specific p-value is an approximation based upon a standard normal distribution.

----Likelihood Ratio Statistic---->
      Global  Degrees Of            
   Statistic     Freedom     P Value
------------  ----------  ----------
    37.11573          39     0.55607

*********************************************************
Marker-Marker Multiallelic D' (bounded between 0 and 1)

                1            2            3            4            5            6
     ------------ ------------ ------------ ------------ ------------ ------------
1               .      0.08708      0.02262      0.05985      0.29364      0.07402
2               .            .      0.03520      0.04269      0.04971      0.07372
3               .            .            .      1.00000      0.04409      0.01931
4               .            .            .            .      0.10358      0.37449
5               .            .            .            .            .      0.31406
6               .            .            .            .            .            .



*********************************************************
Marker-Marker r^2 (Biallelic Markers Only)

                1            2            3            4            5            6
     ------------ ------------ ------------ ------------ ------------ ------------
1               .      0.00655      0.00025      0.00108      0.00130      0.00409
2               .            .      0.00070      0.00013      0.00004      0.00350
3               .            .            .      0.12646      0.00083      0.00014
4               .            .            .            .      0.00055      0.00601
5               .            .            .            .            .      0.00110
6               .            .            .            .            .            .
\end{verbatim}

If DANDELION is executed with a personal probability option, DANDELION writes
the personal probability results to the personal probability output file.  The
personal probability output file contains a list of each individual with all
possible haplotypes pairs in the marker range given that individual's genotype
and the probability that the individual will have that haplotype pair.  See the
example below.

\begin{verbatim}

*********************************
Begin Marker:           1
End Marker:             5
*********************************

DISPLAY PERSONAL PROBABILITIES: CASES
------------------------------------------------------------------

                   Affection
Individual ID      Status       Haplotype #1      Haplotype #2      Probability
-----------------  ---------    --------------    --------------    -----------
1-1309                     2     4  3  2  2  3     2  3  2  2  1         1.0000
1-2                        2     4  1  2  3  3     4  1  2  3  3         1.0000
1-261                      2     4  1  3  3  3     4  1  3  3  3         1.0000
10-10401                   2     4  1  2  3  3     2  1  3  3  3         0.5599
  .
  .
  .

DISPLAY PERSONAL PROBABILITIES: CONTROLS
------------------------------------------------------------------

                   Affection
Individual ID      Status       Haplotype #1      Haplotype #2      Probability
-----------------  ---------    --------------    --------------    -----------
10171744-10171744          1     4  1  2  3  3     2  1  3  3  3         0.7360
10171744-10171744          1     4  1  3  3  3     2  1  2  3  3         0.2640
10531774-10531774          1     4  1  2  3  3     4  3  2  3  3         1.0000
10664238-10664238          1     4  1  2  3  3     4  1  2  3  3         1.0000
11083364-11083364          1     4  1  2  3  3     4  1  3  3  3         1.0000
\end{verbatim}

%% End dandelion.tex \\