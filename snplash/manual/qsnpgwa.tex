%% Most of this discussion is inserted from the document
%% 'qsnpgwa2.1UserGuideDraft.doc' documenting version 2.1
%% Downloaded from the PHS web site 06-Sep-2011

\section{QSNPGWA}
\label{sec:qsnpgwa}
\subsection{Description}

QSNPGWA is a program that reads diallelic marker data and quantitative data for
individuals from input files and computes ardy-Weinberg, genotypic association
and linkage disequilibrium tests.  It also performs linkage disequilibrium tests
for each consecutive pair of markers.

QSNPGWA implements the Expectation-Maximization (EM) Algorithm to calculate two
marker haplotype frequencies for the linkage disequilibrium tests.

QSNPGWA allows a maximum of two integers to represent the alleles of a dialleic
marker in the input file and only allows integers between 1 and 30 to represent
alleles.

\subsubsection{Hardy-Weinberg Proportions Tests}

For each marker, SNPGWA calculates the maximum likelihood estimate of the allele
frequencies, observed and expected number of heterozygotes and homozygotes under
the assumption of Hardy-Weinberg equilibrium and p-values for the chi square
goodness of fit and exact tests for Hardy-Weinberg equilibrium (Guo, 1992)

\begin{enumerate}

\item{Goodness of Fit Test}
See the discussion under SNPGWA, p.\pageref{goodness-of-fit}

\item{Exact Test}
See the discussion under SNPGWA, p.\pageref{exact-test}

\end{enumerate}

\subsubsection{Genotype Tests}
For each marker, QSNPGWA calculates the p-values for four tests of genotypic
association.  Specifically, it calculates the two degree of freedom genotypic
test of association and genotypic tests of association for three genetic models:
dominant, additive and recessive.  For the dominant, additive and recessive
models, QSNPGWA calculates the beta and standard error.  QSNPGWA also calculates
the lack of fit test for the additive genetic model.

With respect to the genotypic tests according to the dominant, additive and
recessive models, QSNPGWA distinguishes between reference (a) and non-reference
(A) alleles for each marker.  The minor allele is the reference allele by
default.  QSNPGWA has an option to set the major allele as the reference allele
(See Section III.4 Map File Option). \{FIX cross-references\} QSNPGWA will not
calculate genotypic tests for monomorphic markers.

QSNPGWA also calculates the mean of the numeric data for groups of reference
allele homozygotes (aa), non-reference allele homozygotes (AA), heterozygotes
(Aa), combined reference allele homozygotes and heterozygotes (aa \& Aa) and
combined non-reference allele homozygotes and heterozygotes (AA \& Aa).

\begin{enumerate}

\item{Two Degree of Freedom Test}


  SNPGWA calculates a two degree of freedom test p-value for each marker.  The
  two degree of freedom test is a one-way ANOVA F-test with the numeric data for
  each marker and each individual assigned to one of three groups: reference
  homozygotes, non-reference homozygotes and heterozygotes.

  The ANOVA is a procedure for testing the null hypothesis that the mean between
  groups are equivalent against the alternative hypothesis that the mean for at
  least two groups are not equivalent (Draper and Smith, 1998).
  Non-equivalent means between groups of reference homozygotes, non-reference
  homozygotes and heterozygotes might be indicative of an underlying genetic
  association.

  The ANOVA procedure is described here.  The within sum of squares term is
  calculated by:

  \begin{equation*}
    WSS = \sum^k_{i=1}\sum^{n{i}}_{j=1}\left(y_{ij} - \bar{y}_i\right)^2
  \end{equation*}

  where $y_{ij}$ is the numeric data for the ith group and the jth individual
  and $\bar{y}_i$ is the mean of the numeric data for the ith group.

  The between sum of squares term is calculated by:

  \begin{equation*}
    BSS = \sum^k_{i=1}\sum^{n{i}}_{j=1}\left(\bar{y}_i - \bar{\bar{y}}\right)^2    
  \end{equation*}

  where $\bar{y}_i$ is the mean of the numeric data for the ith group and
  $\bar{\bar{y}}$ is the mean of numeric data for all groups.

  The test statistic is calculated by:

  \begin{equation*}
    F = \frac{BSS/\left(k-1\right)}{WSS/\left(n-k\right)}
  \end{equation*}

  where $BSS$ and $WSS$ and are calculated above, is the total number of
  individuals and is the number of groups. $k=3$ for the two degree of freedom
  test.

  $F$ is is distributed by an F distribution with $k-1$ and $n-k$ degrees of
  freedom under the null hypothesis.  QSNPGWA calculates the p-value by
  $Pr\left(F_{k-1,n-k} > F\right)$.

  \item{Dominant and Recessive Tests}

    QSNPGWA calculates p-values for each marker using a one-way ANOVA F test
    procedure according to both dominant and recessive models.  Under the
    dominant model, QSNPGWA assigns each individual to either a group of
    non-reference homozygotes or the combined group of reference homozygotes and
    heterozygotes.  Under the recessive model, QSNPGWA assign each individual to
    either a group of reference homozygotes or the combined group of
    non-reference homozygotes and heterozygotes.  QSNPGWA calculates F
    statistics and p-values for the dominant and recessive tests with the
    numeric data for each individual using the same procedure as described for
    the two degree of freedom test.  (See section II.1 Two Degree of Freedom
    Test). \{FIX cross-references\}

    QSNPGWA calculates regression slope (beta) and standard error for each
    marker using regression analysis according to both dominant and recessive
    models.  Under the dominant model, QSNPGWA assigns the following numeric
    values to each individual’s genotype for that marker: 1.0 to reference
    homozygotes, 1.0 to heterozygotes and 0.0 to non-reference homozygotes.
    Under the recessive model, QSNPGWA assigns the following numeric values to
    each individual’s genotype for that marker: 1.0 to reference homozygotes,
    0.0 to heterozygotes and 0.0 to non-reference homozygotes.  QSNPGWA
    calculates regression slope and standard error using the numeric data for
    each individual with the procedure described under the additive test (See
    II.C.3 Additive Test).  \{FIX cross-references\}

  \item{Additive Test}

    The additive test is set up as a linear regression between marker genotypes
    and numeric data.  QSNPGWA calculates regression slope (beta), standard
    error and a p-value for each marker over the individuals in the input file.
    To perform the regression analysis on a marker, QSNPGWA assigns the
    following numeric values to each individual’s genotype for that marker: 1.0
    to reference homozygotes, 0.0 to heterozygotes and -1.0 to non-reference
    homozygotes.  Using the method of least squares (Draper and Smith, 1998),
    QSNPGWA calculates regression slope and standard error with the procedure
    described below where represents the genotype value for a given marker and
    represents the numeric data and is the number of individuals in our sample.

    Corrected sum of squares for x:

    \begin{equation*}
      L_{xx} = \sum^n_{i=1}\left(x_i - \bar{x}\right)^2 = \sum^n_{i=1}x^2_i - \left(\sum^n_{i=1}x_i\right)^2/n
    \end{equation*}

    Corrected sum of squares for y:

    \begin{equation*}
      L_{yy} = \sum^n_{i=1}\left(y_i - \bar{y}\right)^2 = \sum^n_{i=1}y^2_i - \left(\sum^n_{i=1}y_i\right)^2/n
    \end{equation*}

    Corrected sum of cross product:

    \begin{equation*}
      L_{xy} = \sum^n_{i=1}\left(x_i-\bar{x}\right)\left(y_i-\bar{y}\right) = \sum^n_{i=1}x_iy_i - \left(\sum^n_{i=1}x_i\right)\left(\sum^n_{i=1}y_i\right)/n
    \end{equation*}

    Residual mean square:

    \begin{equation*}
      s^2_{y\cdot{}x} = \left(L_{yy} - L^2_{xy}/L_{xx}\right)/\left(n-2\right)
    \end{equation*}

    Regression slope (beta):

    \begin{equation*}
      b = L_{xy}/L_{xx}
    \end{equation*}

    Standard error of b:

    \begin{equation*}
      se(b) = s_{y\cdot{}x}/\left(L_{xx}\right)^{1/2}
    \end{equation*}

  \item{Lack of Fit Test}
    
    QSNPGWA tests the significance of the fit between the additive model
    regression slope and the data (Draper and Smith, 1998).
    QSNPGWA calculates lack of fit and pure error between the regression slope
    and the test data.  Using these results, QSNPGWA calculates the lack of fit
    F statistic and p-value using an ANOVA procedure.

\end{enumerate}

\subsubsection{Linkage Disequilibrium Tests}
See discussion under SNPGWA, p.\pageref{sec:linkage-dis}

\subsection{Usage}
See ``SNPlash'', p.\pageref{sec:snplash}, for options common to all engines.

\begin{verbatim}

  snplash -engine qsnpgwa -geno <filename> -phen <filename> \
          -out <filename> -map <filename> [OPTIONS]

where options are

  -cov            Comma-separated names of covariates from the
                  phenotype file
  --geno_file     Print extra genotypic data to <outfile>.geno[1,2,3]
  --haplo_file    Print additional haplotype information to
                  <out>.haplo[1,2,3]
  --hwe_file      Print additional Hardy-Weinberg statistics to
                  <out>.hwe[cntrl,case]
  --haplo_thresh  Set the threshold number of chromosomes across
                  all participants on which a haplotype must 
                  exist to be used for global haplotype testing.
  --val           Print additional statistics to <out>.statvals

\end{verbatim}

\subsection{Output}
General, Hardy-Weinberg, association and linkage statistics are printed.  The
specific  variables are:

\vspace{1em}
\begin{supertabular}{r p{5in}}
  General & xx ~ \\
    Chr: & The chromosomal location for the feature \\
    Marker: & The marker label \\
    Position: & The position of the feature on the chromosome \\
    Difference: & xx \\
    Total Number Indivs: & xx \\
    Minor Allele Freq: & Minor allele frequency\\
    \%{}\_Miss: & Percentage records missing a value for this allele\\
    \%{}\_Miss PVal: & xx \\
    P: & xx \\
    Q: & xx \\
    Ref(a): & xx \\
    ~ & xx ~\\[1em]
  Hardy-Weinberg Analysis & xx ~ \\
    NumPP: & xx    \\
    ENumPP: & xx    \\
    NumPQ: & xx    \\
    ENumPQ: & xx    \\
    NumQQ: & xx    \\
    ENumQQ: & xx    \\
    X\^2\_PValue: & xx    \\
    Prob\_HWE: & xx     \\
    ~ & xx ~ \\
  Quantitative Genotypic Association & xx \\
          Two-degree-of-freedom test & xx \\
   PV\_2DF: & P-value for the two-degree-of-freedom test \\
   ~ & xx ~ \\[1em]
  Dominant Test & xx ~ \\
      PV\_Dom: & P-value \\
      Dom\_Beta: & Coefficient \\
      Dom\_SE: & Standard error \\[1em]

   Additive Test & xx ~ \\
      PV\_Dom: & P-value \\
      Dom\_Beta: & Coefficient \\
      Dom\_SE: & Standard error \\[1em]

   Recessive Test & xx ~ \\
      PV\_Dom: & P-value \\
      Dom\_Beta: & Coefficient \\
      Dom\_SE: & Standard error \\[1em]

   Lack of Fit & xx ~ \\
      PV\_LOF: & Lack of fit p-value \\[1em]

   Means and Standard Deviations & xx ~ \\
      Mean\_AA: & xx \\
      SD\_AA: & xx \\
      Mean\_Aa: & xx \\
      SD\_Aa: & xx \\
      Mean\_aa: & xx \\
      SD\_aa: & xx  \\
      Mean\_AA\&Aa: & xx  \\
      SD\_AA\&Aa : & xx \\
      Mean\_Aa\&aa: & xx  \\
      SD\_Aa\&aa: & xx  \\ [1em]

   (Linkage) & xx ~ \\
      D\_Prime: & xx $D'$ \\
      R\_Sqr: & $r^2$  \\
 
\end{supertabular}

%% End qsnpgwa.tex \\ 
