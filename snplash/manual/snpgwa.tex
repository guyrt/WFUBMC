% This section was copied from the document 'SNPGWA4.0_User_Guide.doc'
% downloaded from the PHS web site 06-Sep-2011
\section{SNPGWA}
\label{sec:snpgwa}
\subsection{Description}
Logistic regression with a binomial trait (case/control) and optional
covariates.  SNPGWA allows a maximum of two alphanumeric characters per genetic
marker to represent the alleles of a diallelic marker in the input file.
Missing allelic data is represented by 0 (zero) and does not count toward the
maximum of two characters.

\subsubsection{Hardy-Weinberg Proportions Tests}
For each marker, SNPGWA calculates the maximum likelihood estimate of the allele
frequencies, observed and expected number of heterozygotes and homozygotes under
the assumption of Hardy-Weinberg equilibrium and p-values for the chi square
goodness of fit and exact tests for Hardy-Weinberg equilibrium \cite{Guo92}.

\subsubsection{Goodness of Fit Test}
\label{goodness-of-fit}
SNPGWA calculates p-values for the Hardy-Weinberg goodness of fit test for the
combined group of cases and controls only.  The table and equations below
describe how SNPGWA categorizes single marker allele data and calculates the chi
square value for the HW goodness of fit test.

\vspace{1em}
\begin{tabular}{rccc}
  \hline
  {} & \textbf{AA} & \textbf{Aa} & \textbf{aa} \\
  \hline
  Case and Controls & $n_1$ & $n_2$ & $n_3$ \\
  \hline
\end{tabular}
\vspace{1em}

SNPGWA calculates allele frequencies (\textit{p} and \textit{q}), expected
numbers of homozygotes and heterozygotes ($E_1$, $E_2$ and $E_3$) and chi square
value ($\chi^2$) for the goodness-of-fit test according to the following
equations.

\begin{equation*}
  p = \frac{2n_1 + n_2}{2n_1 + 2n_2 + 2n_3}, \quad  q = \frac{2n_3 +n_2}{2n_1 + 2n_2 + 2n_3}
\end{equation*}

\begin{equation*}
  E_1=p^2(n_1 + n_2 +n_3), E_2 = 2pq(n_1 + n_2 +n_3) \text{ and } E_3=(n_1 + n_2 +n_3)
\end{equation*}

\begin{equation*}
  \chi^2 = \frac{(E_1 - n_1)^2}{E_1} + \frac{(E_2 - n_2)^2}{E_2} + \frac{(E_2 - n_3)^2}{E_3}
\end{equation*}

$\chi^2$ is distributed as an asymptotic chi square value with one degree of
freedom.  Low p-values indicate deviation from Hardy-Weinberg Equilibrium.

\subsubsection{Exact Test}
\label{exact-test}
SNPGWA calculates p-values for the Hardy-Weinberg exact test for the combined
group of cases and controls, cases only and controls only.  The exact test is
that of Wigginton et al. and is calculated as follows \cite{Wiggington05}.

Consider a sample of $N$ individuals and a diallelic marker with $n_A$ alleles of
type $A$ and $n_{Aa}$ heterozygotes.  Note that if $N$, $n_A$ and $n_{Aa}$ are
known then $n_a$, $n_{aa}$ and $n_{AA}$ can be determined.  Calculating the
exact test p-value begins by calculating the probability of every possible
heterozygote configuration given $N$ and $n_a$ ($P(N_{Aa} = n_{Aa} |N,n_A)$ for
all $n_{Aa}$). Once these are known, the p-value is the sum of the probability
of all heterozygote configuration that are less than the probability of the
observed heterozygote configuration in our sample.
\begin{equation*}
(P_{HWE}=\sum_{n^*_{Aa}}I[P(N_{Aa} = n_{Aa}|N,n_a)\geq P(N_{Aa} = n^*_{Aa}|N,n_A)] * P(N_{Aa} = n^*_{Aa}|N,n_A)
\end{equation*}

\vspace{1em}
where $I[x]$ is an indicator function that is equal to 1 when the
comparison is true and 0 when false).  SNPGWA implements an iterative process
for calculating heterozygote probabilities (Guo and Thompson, \cite{Guo92}) by
determining the heterozygote configuration with the highest probability and
assigning it a probability of 1.0.  All other heterozygotes probabilities are
calculated according to the formulas:
\vspace{1em}

\begin{equation*}
  P(N_{Aa} = n_{Aa}+2|N,n_A) = P(N_{Aa}=n_{Aa} |N,n_A)\frac{4n_{AA}n_{aa}}{(n_{Aa}+2)(n_{Aa}+1)}
\end{equation*}

\noindent{}and

\begin{equation*}
  P(N_{Aa}=n_{Aa}-2|N,n_A) = P(N_{Aa}=n_{Aa}|N,n)\frac{4n_{Aa}(n_{Aa}-1)}{4(n_{AA}+1)(n_{aa}+1)}.
\end{equation*}

\noindent{}SNPGWA then calculates the p-value as described above.  Low p-values indicate
deviation from Hardy-Weinberg Equilibrium.

\subsubsection{Genotype Tests}
For each marker, SNPGWA calculates the p-values for four tests of genotypic
association.  Specifically, it calculates the two degree of freedom genotypic
test of association and tests for three genetic models: dominant, additive and
recessive.  SNPGWA also calculates the lack of fit test for the additive genetic
model.  SNPGWA calculates the odds ratio and odds ratio 95\% confidence interval
for the dominant, additive and recessive tests.  SNPGWA allows the option of
calculating empirical p-values for all genotypic tests.  See \{FIX internal
refs\} Sections I.G: Empirical P-Value Calculation and III.A.4: Empirical
P-Value Calculation from the Command Line for information on empirical p-value
calculation.

The genotype tests may be performed using covariate adjustment by specifying
covariates (which must be included in the phenotype file) on the command line
via the -cov option.  See Section III.A.  \{FIX internal refs\}

\begin{enumerate}
\item{Two Degree of Freedom Test}
SNPGWA calculates the two degree of freedom genotypic association test as the
standard 2 x 3 contingency table chi square test of independence (Agresti
\cite{Agresti02}).  Low p-values indicate deviation from independence.

The tables below describe how SNPGWA categorizes single marker allele data and
calculates the chi square value for the two degree of freedom test.

\begin{center}
  \begin{tabular}{lcccc}
    \hline
    {}     & \textbf{AA} & \textbf{Aa} & \textbf{aa} & \\
    \hline
    Case    & $n_{11}$ & $n_{12}$ & $n_{13}$ & $n_{1+}$ \\
    Control & $n_{21}$ & $n_{22}$ & $n_{23}$ & $n_{2+}$ \\
    {}    & $n_{+1}$ & $n_{+2}$ & $n_{+3}$ & $n$     \\
    \hline
    \label{tab:n_tab}
  \end{tabular}
\end{center}

\begin{center}
  \begin{tabular}{lccc}
    \hline
    {} & \textbf{AA} & \textbf{Aa} & \textbf{aa} \\
    \hline
    Case    & $E_{11} = n_{1+}n_{+1}/n$ & $E_{12} = n_{1+}n_{+2}/n$ & $E_{13} = n_{1+}n_{+3}/n$ \\
    Control & $E_{21} = n_{2+}n_{+1}/n$ & $E_{22} = n_{2+}n_{+2}/n$ & $E_{23} = n_{2+}n_{+3}/n$ \\
    \hline
    \label{tab:e_tab}
  \end{tabular}
\end{center}

\begin{align*}
  \chi^2 &= \frac{(E_{11}-n_{11})^2}{E_{11}} + \frac{(E_{12}-n_{12})^2}{E_{12}} + \frac{(E_{13}-n_{13})^2}{E_{13}} \\
         &+ \frac{(E_{21}-n_{21})^2}{E_{21}} + \frac{(E_{22}-n_{22})^2}{E_{22}} + \frac{(E_{23}-n_{23})^2}{E_{23}}
\end{align*}

\noindent{}$\chi^2$ is distributed as an asymptotic chi square value with two
degrees of freedom.

\item{Dominant and Recessive Tests}
The tests for the dominant and recessive genetic models are set up as a $2 * 2$
contingency table chi square test, combining heterozygotes and homozygotes.  The
dominant model tests for an association between the outcome and having at least
one copy of the reference allele, while the recessive model tests for an
association between the outcome and having two copies of the reference allele.
By default, SNPGWA assigns the minor allele as the reference allele.  SNPGWA
allows the option to specify the reference allele for the dominant and recessive
tests from the map file.  See Section III.A.5, Map File Option.  Note that if
the reference allele for a single marker is switched from the minor allele to
the major allele then the dominant and recessive test p-values, odds ratios and
95\% confidence intervals are reversed in the output file(s).

The tables below describe how SNPGWA categorizes single marker allele data and
calculates the chi square value for the dominant and recessive tests.  Lower
case 'a' represents the reference allele.

\vspace{2em}
Dominant Test:

\begin{center}
  \begin{tabular}{lccc}
    \hline
    {}  & \textbf{AA}  & \textbf{Aa \& aa} & {} \\
    \hline
    Case    & $n_{11}$ & $n_{12} + n_{13}$ & $n_{1+}$ \\
    Control & $n_{21}$ & $n_{22} + n_{23}$ & $n_{2+}$ \\
    {}    & $n_{+1}$ & $n_{+2}$         & $n$ \\
    \hline  
  \end{tabular}
\end{center}

\begin{center}
  \begin{tabular}{lcc}
    \hline
    {}     & \textbf{AA}            & \textbf{Aa \& aa} \\
    \hline
    Case    & $E_{11} = n_{1+}n_{+1}/n$ &  $E_{12} = n_{1+}n_{+2}/n$ \\
    Control & $E_{21} = n_{2+}n_{+1}/n$ &  $E_{22} = n_{2+}n_{+2}/n$ \\
    \hline
  \end{tabular}
\end{center}

\vspace{2em}
Recessive Test:

\vspace{2em}
\begin{center}
  \begin{tabular}{lccc}
    \hline
    {}     & \textbf{AA \& aa} & \textbf{aa} & {} \\
    \hline
    Case    & $n_{11} + n_{12}$   & $n_{13}$ & $n_{1+}$ \\
    Control & $n_{21} + n_{22}$   & $n_{23}$ & $n_{2+}$ \\
    {}     & $n_{+1}$           & $n_{+2}$ & $n$ \\
    \hline  
  \end{tabular}
\end{center}

\begin{center}
  \begin{tabular}{lcc}
    \hline
    {}     & \textbf{AA \& Aa}    & \textbf{aa} \\
    \hline
    Case    & $E_{11}=n_{1+}n_{+1}/n$ & $E_{12}=n_{1+}n_{+2}/n$ \\
    Control & $E_{21}=n_{2+}n_{+1}/n$ & $E_{22}=n_{2+}n_{+2}/n$ \\
    \hline
  \end{tabular}
\end{center}

\begin{equation*}
  \chi^2 = \frac{(E_{11}-n_{11})^2}{E_{11}} + \frac{(E_{12}-n_{12})^2}{E_{12}} +\frac{(E_{21}-n_{21})^2}{E_{21}} + \frac{(E_{21}-n_{21})^2}{E_{21}}
\end{equation*}

$\chi^2$ for both the dominant and recessive tests is distributed
as an asymptotic chi square value with one degree of freedom.

SNPGWA calculates the odds ratios and 95\% confidence intervals for the dominant
and recessive genetic models from the $2 x 2$ tables above (Agresti 2002
\cite{Agresti02}):

Dominant Test:

\begin{equation*}
  OR = \frac{n_{11}/(n_{12}+n_{13})}{n_{21}/(n_{22}+n_{23})}
\end{equation*}

\begin{equation*}
  (LCI,UCI) = (\exp(log(OR) - (1.96B)), \exp(log(OR)+1.96B))
\end{equation*}

where 

\begin{equation*}
  B = \sqrt{\frac{1}{n_{11}} + \frac{1}{n_{12}+n_{13}} + \frac{1}{n_{21}} + \frac{1}{n_{22}+n_{23}}}
\end{equation*}

SNPGWA calculates sensitivity, specificity and C-statistics for dominant and
recessive models (Rosner, 2006 \cite{Rosner06}).  The C-statistic is the area
under the receiver operating characteristic (ROC) curve.

Using the tables above, sensitivity, specificity and C-statistic under the
dominant model are calculated as follows.

\begin{equation*}
  Sens = (n_{12} + n_{13})/n_{1+}
\end{equation*}

\begin{equation*}
  Spec = n_{11}/n_{1+}
\end{equation*}

\begin{equation*}
  C\-Stat = \frac{1}{2}(Sens)(1-Spec) + \frac{1}{2}(1-Sens)(1-(1-Spec))=\frac{1}{2}(Sens+Spec)
\end{equation*}

Similarly, sensitivity, specificity and C-statistic under the recessive model are

\begin{equation*}
  Sens = n_{13}/n_{1+}
\end{equation*}

\begin{equation*}
  Spec = (n_{11} +n_{12})/n_{1+}
\end{equation*}

\begin{equation*}
  C\-Stat = \frac{1}{2}(Sens+Spec)
\end{equation*}

\item{Additive Test}

SNPGWA implements the Cochran-Armitage test for trends as the test for the
additive genetic model.  The calculation of the Cochran-Armitage test and
goodness of fit test are described below.

Let the table below contain the genotype data for a single diallelic marker.
Lower case 'a' represents the reference allele.  By default, SNPGWA assigns the
minor allele as the reference allele.  SNPGWA allows the option to specify the
reference allele for the additive test from the map file.  See \{FIX links\}
Section III.A.5, Map File Option.

\begin{center}
  \begin{tabular}{lcccc}
    \hline
    {} & \textbf{AA} & \textbf{Aa} & \textbf{aa} & {} \\
    \hline
    Case & $n_{11}$ & $n_{12}$ & $n_{13}$ & $n_{1+}$ \\
    Control & $n_{21}$ & $n_{22}$ & $n_{23}$ & $n_{2+}$ \\
    {}   & $n_{+1}$ & $n_{+2}$ & $n_{+3}$  & $n$ \\
    \hline
  \end{tabular}
\end{center}

SNPGWA calculates the p-value for the Cochran Armitage test for trends as
follows (Agresti 2002, \cite{Agresti02}):

\begin{center}
  \begin{tabular}{ll}
    $\bar{X_1} = (1n_{+1})/n$ &  $BD_1 = n_{+1}(1-sum\bar{X})^2$ \\
    $\bar{X_2} = (2n_{+2})/n$ &  $BD_2 = n_{+2}(1-sum\bar{X})^2$ \\
    $\bar{X_3} = (3n_{+3})/n$ &  $BD_3 = n_{+3}(1-sum\bar{X})^2$ \\
    $sum\bar{X} = \sum_{i=1}^3\bar{X}_i$ & $sumBD = \sum_{i=1}^3\overline{BD}_i$ \\
  \end{tabular}
\end{center}
\vspace{1em}

\begin{center}
  \begin{tabular}{c}
    $BN_1 = n_{+1}(n_{11}/n_{+1} -n_{1+}/n)(1-sum\bar{X})$ \\
    $BN_2 = n_{+2}(n_{12}/n_{+2} -n_{1+}/n)(2-sum\bar{X})$ \\
    $BN_3 = n_{+3}(n_{13}/n_{+3} -n_{1+}/n)(3-sum\bar{X})$ \\
  \end{tabular}
\end{center}

\begin{equation*}
  sumBN = \sum_{i=1}^3BN_i
\end{equation*}

\vspace{1em}
\begin{equation*}
  B = sumBN/sumBD
\end{equation*}

\begin{equation*}
  \chi^2 = \left(\frac{B^2}{n_{1+}/n(1-n_{1+}/n)}\right)sumBD
\end{equation*}

$\chi^2$ is distributed as an asymptotic chi square value with one degree of freedom.

SNPGWA calculates the odds ratio and 95\% confidence intervals for the additive
test by a two parameter logistic regression model (Agresti 2002,
\cite{Agresti02}).  The logit is expressed in the following form:

\vspace{1em}
\begin{equation*}
  g(\mathbf{x}) = \beta_0 + \beta_1x_1
\end{equation*}

The regression coefficients $\beta_0$ and $\beta_1$ are calculated using a
Newton-Raphson iterative method (Agresti 2002 \cite{Agresti02}) as described
below.  SNPGWA makes three attempts to calculate regression coefficients with
regression coefficients set to different initial conditions on each attempt.
The regression coefficients are initially set to 0.5, 0.0 and -0.5 on each
successive attempt.  If the Newton-Raphson method does achieve convergence to a
solution on the three attempts, SNPGWA sets the odds ratio and upper and lower
confidence interval bounds to 2.0.

Each iteration of the Newton-Raphson method begins with calculating
$\pi_i^{(t)}$ for each individual $i$.

\begin{equation*}
  \pi_i^{(t)} = \frac{\exp(\beta_0 + \beta_1x_{i1})}{1+\exp(\beta_0 + \beta_1x_{i1})}
\end{equation*}

In accordance with the Cochran-Armitage Trends test implemented in SNPGWA, with
respect to a specific SNP, each individual falls into one of three categories:
non-reference allele homozygotes $(x_{i1}=1)$ , heterozygotes $(x_{i1}=2)$ and
reference allele homozygotes $(x_{i1}=3)$ .

The regression coefficients for the next iteration are calculated according to
the Newton-Raphson formula:

\begin{equation*}
  \boldsymbol{\beta}^{(t+1)} = \boldsymbol{\beta}^{(t)} + \left\{\mathbf{X'diag}[n_i\pi_i^{(t)}(1-\pi_i^{(t)})]\mathbf{X}\right\}^{-1}\mathbf{X'}(\mathbf{y}-\mathbf{u}^{(t)})
\end{equation*}

where $\mu_i^{(t)} = n_i\pi_i^{(t)}$ and $n_i$ are the number of incidences of
individuals with duplicate $x_{i1}$ values.  The intermediates in the
Newton-Raphson formula take the following values:

\begin{equation*}
  \mathbf{X} =
  \begin{pmatrix}
    1 & x_{11} \\
    1 & x_{12} \\
    1 & x_{13} \\
  \end{pmatrix}
  =
  \begin{pmatrix}
    1 & 1 \\
    1 & 2 \\
    1 & 3 \\
  \end{pmatrix}
\end{equation*}

\begin{equation*}
  \mathbf{diag}[n_i\pi_i^{(t)}(1-\pi_i^{(t)})] = 
  \begin{pmatrix}
    (n_{11}+n_{21})\pi_1^{(t)}(1-\pi_1^{(t)}) & 0 & 0 \\
    0 & (n_{12}+n_{22})\pi_2^{(t)}(1-\pi_2^{(t)}) & 0 \\
    0 & 0 & (n_{13}+n_{23})\pi_3^{(t)}(1-\pi_3^{(t)}) \\
  \end{pmatrix}
\end{equation*}

\begin{equation*}
  (\mathbf{y}-\boldsymbol{\mu}^{(t)}) = 
  \begin{pmatrix}
    n_{11} - (n_{11} + n_{21})\pi_1^{(t)} \\
    n_{12} - (n_{12} + n_{22})\pi_2^{(t)} \\
    n_{13} - (n_{13} + n_{23})\pi_3^{(t)} \\
  \end{pmatrix}
\end{equation*}

Once SNPGWA calculates the regression coefficients, the odds ratio and 95\%
confidence interval are calculated as follows:

\noindent{}$OR = \exp{\beta_1}$

\begin{equation*}
  (LCI, UCI) = \left(\exp\left(\beta_1 - 1.96\sqrt{B}\right), \exp\left(\beta_1 + 1.96\sqrt{B}\right)\right)
\end{equation*}

where $B$ is the diagonal element of the inverse information matrix
corresponding to $\beta_1$ .

Under the additive model, SNPGWA calculates sensitivity and specificity between
non-reference allele homozygotes and heterozygotes, between non-reference allele
homozygotes and reference allele homozygotes and between heterozygotes and
reference allele homozygotes.  SNPGWA also calculates the C-statistic under the
overall additive model. (Rosner, 2006 \cite{Rosner06})

Using the table above, sensitivities and specificities are calculated as
follows.

\{FIX: This equation set has two definitions for Sens and Spec AA/aa\}
\begin{center}
  \begin{tabular}{c}
    $Sens_{AA/Aa} = n_{12}/(n_{11} + n_{12}$ \\
    $Spec_{AA/Aa} = n_{21}/(n_{21} + n_{22}$ \\
    $Sens_{AA/aa} = n_{13}/(n_{11} + n_{13}$ \\
    $Spec_{AA/aa} = n_{21}/(n_{21} + n_{23}$ \\
    $Sens_{AA/aa} = n_{13}/(n_{12} + n_{13}$ \\
    $Spec_{AA/aa} = n_{22}/(n_{22} + n_{23}$ \\
  \end{tabular}
\end{center}

The C-statistic is the area under the receiver operating characteristic (ROC)
curve.  The C-statistic under the additive model is calculated as follows.

\begin{align*}
  \text{C-Stat} &= \frac{1}{2}\left(\frac{n_{23}}{n_{2+}}\right)\left(\frac{n_{13}}{n_{1+}}\right) \\
          &+ \frac{1}{2}\left(\frac{n_{22}}{n_{2+}}\right)\left(2\left(\frac{n_{13}}{n_{1+}}\right)+\left(\frac{n_{12}}{n_{1+}}\right)\right) \\
          &+ \frac{1}{2}\left(\frac{n_{21}}{n_{2+}}\right)\left(2\left(\frac{n_{13}}{n_{1+}}\right) + 2\left(\frac{n_{12}}{n_{1+}}\right) + \left(\frac{n_{11}}{n_{1+}}\right)\right)
\end{align*}

\item{Lack of Fit Test} 

  SNPGWA calculates the lack of fit p-value as follows (\cite{Agresti02}), ($B$
  and $sum\bar{X}$ are intermediates from the additive test calculation in the
  previous section):
  \begin{center}
    \begin{tabular}{ll}
      $\pi_1 = n_{+1}/n + B(1-sum\bar{X})$ & $C_1 = n_{+1}(n_{11}/n_{+1} - \pi_1)^2$ \\
      $\pi_2 = n_{+1}/n + B(2-sum\bar{X})$ & $C_2 = n_{+2}(n_{12}/n_{+2} - \pi_2)^2$ \\
      $\pi_3 = n_{+1}/n + B(3-sum\bar{X})$ & $C_3 = n_{+3}(n_{13}/n_{+3} - \pi_3)^2$ \\
    \end{tabular}
  \end{center}

  \begin{equation*}
    sumC = \sum_{i=1}^3C
  \end{equation*}
  \begin{equation*}
    \chi^2 = \left(\frac{1}{n_{+1}/n(1-n_{1+}/n)}\right)sumC
  \end{equation*}

$\chi^2$ is distributed as an asymptotic chi square value with one degree of freedom.

\vspace{1em}

\item{Low Expected Values}

  SNPGWA calculates expected numbers of case and control individuals for each
  genotype in the standard $2x3$ table for each genotype
  ($E_{11}$, $E_{12}$, $E_{13}$, $E_{21}$, $E_{22}$, $E_{23}$ in the table on page
  \pageref{tab:e_tab}).  These are necessary intermediates for calculating
  p-values for the two degree of freedom test.  In the case that an expected
  value is low enough to induce an unusually high chi square and subsequent low
  p-value, SNPGWA adds 0.5 to the number of case and control individuals for
  each genotype ($n_{11}, n_{12}, n_{13}, n_{21}, n_{22}, n_{23}$ in the table
  on page \pageref{tab:n_tab}).  The genotypic and lack-of-fit p-values are
  calculated using the corrected totals.  SNPGWA uses an expected value
  threshold of 3.0 to induce SNPGWA to recalculating with the continuity
  correction.

  For example, suppose the tables below contain data and expected values for a
  single marker.
  \begin{center}
    \begin{tabular}{lcccc}
      \hline
      {}     & \textbf{AA} & \textbf{Aa} & \textbf{aa} & {}\\
      \hline
      Case    & 120 & 87 & 6 & 213 \\
      Control & 133 &  7 & 1 & 141 \\
      {}    & 253 & 94 & 7 & 354 \\
      \hline
      % \label{}
    \end{tabular}
  \end{center}

  \begin{center}
    \begin{tabular}{lccc}
      \hline
      {}     & \textbf{AA} & \textbf{Aa} & \textbf{aa} \\
      Case    & 152.23      & 56.56       & 4.21 \\
      Control & 100.77      & 37.44       & 2.79 \\
      \hline
      % \label{tab}
    \end{tabular}
  \end{center}

Because the expected number of controls with genotype aa is less that 3.0,
SNPGWA adds 0.5 to the total number of case and control individuals for each
genotype.  See the tables below.

\begin{center}
  \begin{tabular}{lcccc}
    \hline
    {}     & \textbf{AA} & \textbf{Aa} & \textbf{aa} & {} \\
    \hline
    Case    & 120.5 & 87.5 & 6.5 & 214.5 \\
    Control & 133.5 &  7.5 & 1.5 & 142.5 \\
    {}    & 254   & 95   & 8 & 357 \\
    \hline
    % \label{}
  \end{tabular}
\end{center}

\begin{center}
  \begin{tabular}{lccc}
    \hline
    {}     & \textbf{AA} & \textbf{Aa} & \textbf{aa}  \\
    \hline
    Case    & 152.61      & 57.08       & 4.81 \\
    Control & 100.39      & 37.92       & 3.19 \\
    \hline
    % \label{tab}
  \end{tabular}
\end{center}

SNPGWA calculates genotypic tests, odds ratios, 95\% confidence intervals and
the lack-of-fit test with the modified data.

\end{enumerate}

\subsubsection{Linkage Disequilibrium Tests}
\label{sec:linkage-dis}
For each consecutive pair of diallelic markers, SNPGWA calculates values for D
Prime $\left(D'\right)$ and R Squared $\left(R^2\right)$ linkage disequilibrium
tests (Devlin and Risch \cite{Devlin95}).  SNPGWA begins by calculating
autosomal haplotype frequencies for the two consecutive markers using the EM
algorithm (Slatkin and Excoffier \cite{Slatkin96}) and assuming Hardy-Weinberg
equilibrium.  Then $\left(D'\right)$ and $\left(R^2\right)$ are calculated as
follows:
\begin{equation*}
 D = freq_{11}freq_{22} - freq_{12}freq_{21}
\end{equation*}

\begin{equation*}
\text{If } D > 0, D' = \frac{D}{min\left(freq_{1+}freq_{+2}, freq_{+1}freq_{2+}\right)}
\end{equation*}

\begin{equation*}
  \text{If } D < 0, D' = \frac{D}{min\left(freq_{1+}freq_{+1}, freq_{+2}freq_{2+}\right)}
\end{equation*}

\begin{equation*}
  R^2 = \frac{D}{\sqrt{\left(freq_{1+}freq_{2+}freq_{+1}freq_{+2}\right)}}
\end{equation*}

Where $freq_{11}$, $freq_{12}$, $freq_{21}$ and $freq_{22}$ are autosomal
haplotype frequencies as calculated by the EM algorithm and
$freq_{1+} = freq_{11} + freq_{12}$, $freq_{2+} = freq_{21} + freq_{22}$,
$freq_{+1} = freq_{11} + freq_{21}$ and $freq_{+2} = freq_{12} + freq_{22}$.

\subsubsection{Allelic Test}

SNPGWA calculates the allelic association test as a $2 x 2$ contingency table
chi square test of independence (Guedj, Wojcik, Della-Chiesa, Nuel, Forner 2006
\cite{Guedj06}).  Low p-values indicate deviation from independence.

The tables below describe how SNPGWA categorizes single marker allele data and
calculates the chi square value for the allelic test.

\begin{center}
  \begin{tabular}{lccc}
    \hline
    {}     & \textbf{AA} & \textbf{Aa} & \textbf{aa} \\
    \hline
    Case    & $n_{11}$ & $n_{12}$ & $n_{13}$ \\
    Control & $n_{21}$ &  $n_{22}$ & $n_{23}$ \\
    \hline
    % \label{}
  \end{tabular}
\end{center}

\begin{center}
  \begin{tabular}{lccc}
    \hline
    {} & Case & Control & {} \\
    \hline
    A  & $m_{11} = 2n_{11} + n_{12}$ & $m_{12} = 2n_{13} + n_{12}$ & $m_{1+}$ \\
    a  & $m_{21} = 2n_{21} + n_{22}$ & $m_{22} = 2n_{23} + n_{22}$ & $m_{2+}$ \\
    {} & $m_{+1}$                  & $m_{+2}$                   & $m$     \\
    \hline
  \end{tabular}
\end{center}

\begin{center}
  \begin{tabular}{ccc}
    \hline
    {} & Case & Control \\
    \hline
    A  & $E_{11} = m_{1+}m_{+1}/m$ & $E_{21} = m_{1+}m_{+2}/m$ \\
    a  & $E_{12} = m_{2+}m_{+1}/m$ & $E_{22} = m_{2+}m_{+2}/m$ \\  
    \hline
  \end{tabular}
\end{center}


\begin{equation*}
  \chi^2 = \frac{\left(E_{11} - m_{11}\right)^2}{E_{11}} + \frac{\left(E_{12} - m_{12}\right)^2}{E_{12}} + \frac{\left(E_{21} - m_{21}\right)^2}{E_{21}} + \frac{\left(E_{22} - m_{22}\right)^2}{E_{22}}
\end{equation*}

$\chi^2$ is distributed as an asymptotic chi square value with one degree of
freedom.

For each marker, SNPGWA allows empirical p-value calculation for the allelic
test.  See Sections I.G: Empirical P-Value Calculation \{FIX refs\} and III.A.4:
Empirical P-Value Calculation from the Command Line for information on empirical
p-value calculation.


\subsubsection{Two and Three Marker Haplotype Tests}
For each two consecutive markers and three consecutive markers, SNPGWA
calculates a p-value for the likelihood ratio statistic (LRS) (Green, Langefeld
and Lange 2001 \cite{Green01}) to ascertain heterogeneity in the observed
haplotype frequencies between cases and controls.  To calculate the LRS, SNPGWA
must first calculate autosomal haplotype frequencies within each marker, two
consecutive markers and three consecutive markers for three groups: (1) combined
case and control individuals, (2) case individuals and (3) control individuals.
Autosomal haplotype frequencies are calculated using the EM algorithm, assuming
Hardy-Weinberg equilibrium.

The LRS for each marker, two consecutive markers and three consecutive markers
is calculated as follows:

\begin{equation*}
  LRS = 2log\left(\frac{\prod^N_{i=1}{Pr\left(g_i|\tilde{f}_{cases}\right)}\prod^{2N}_{j=N+1}Pr\left(g_j|\tilde{f}_{controls}\right)}{\prod^{2N}_{k=1}Pr\left(g_k|\tilde{f}_{combined}\right)}\right)
\end{equation*}

\noindent{} where $g_i$ represents the multilocus genotypes in the $i^{th}$
individual.

LRS is distributed as a chi square variable with the degrees of freedom set to
the number of haplotypes that have significant autosomal haplotype frequencies
(greater than or equal to 0.0001) in either the case or control groups minus
one.

For example, suppose SNPGWA calculates the LRS for three consecutive diallelic
markers.  In the process SNPGWA employs the EM algorithm to calculate autosomal
haplotype frequencies within the three consecutive markers for groups of case
individuals, control individuals and combined case and control individuals.
Since each marker is diallelic there are eight possible haplotypes that an
individual could possess.  If SNPGWA determines five haplotypes in the case or
control group have significant probability (greater than or equal to 0.0001),
then the LRS is distributed as a chi square variable with 5 – 1 = 4 degrees of
freedom.

For each consecutive two and three markers, SNPGWA allows empirical p-value
calculation for the two and three marker haplotype tests.  See Sections I.G:
Empirical P-Value Calculation and III.A.4: Empirical P-Value Calculation from
the Command Line \{FIX reference \} for information on empirical p-value
calculation.

\subsubsection{Empirical P-Value Calculation}

SNPGWA allows the option of calculating empirical p-values for all genotypic
tests, the allelic test and the two and three marker haplotype tests.  SNPGWA
calculates a chi square value for each test as described in the sections above
using unmodified sample data prior to calculating an empirical p-value for that
test.  This chi square value is necessary for calculating the empirical p-value.
This document refers to it as the unpermuted chi square value
$\left(\chi^2_u\right)$.

SNPGWA calculates an empirical p-value for a test by executing the following
three steps a specified number of times.  This document refers to the number of
times that SNPGWA executes steps 1 through 3 for a specific test as the number
of permutations for that test.

\begin{enumerate}
\item Permute the affection statuses (case or control) of the entire sample
  represented in the input file while preserving the total number of cases and
  total number of controls.

\item Execute the specific test as described in the sections above using the
  permuted sample.

\item Compare the chi square value from the test using permuted data with the
  unpermuted chi square value.  The chi square value using permuted data is
  referred to as the permuted chi square value $\left(\chi^2_{p_{i}}\right)$.

\end{enumerate}

SNPGWA records the number of instances that the permuted chi square exceeds the
unpermuted chi square value.  The permuted chi square value is considered to
exceed the unpermuted chi square value when the permuted chi square value is
greater than the sum of the unpermuted chi square value and a small epsilon.
Epsilon is set to 0.0001 in SNPGWA.  If a permuted chi square value is within
epsilon of the unpermuted chi square value, above or below, then SNPGWA adds 0.5
to the number of instances the permuted chi square value exceeds the unpermuted
chi square value.  The empirical p-value is the number of instances that the
permuted chi square value exceeds the unpermuted chi square value divided by the
total number of permutations.  See the equation below.

\begin{equation*}
  EmpPVal = \frac{\sum^{NumPerms}_{i=1}\left\{I[\chi^2_{p_{i}} > \chi^2_{u} + \epsilon] + \frac{1}{2}I[\chi^2_u + \epsilon \ge \chi^2_{p_{i}} > \chi^2_{u} - \epsilon]\right\}}{NumPerms}
\end{equation*}

Where $I[]$ is an indicator function equal to 1 when the statement in
the bracket is true and $\epsilon = 0.0001$.

SNPGWA allows two options for setting the number of permutations to calculate
empirical p-values: 1) use a fixed number of permutations or 2) use a variable
number of permutations.  If a fixed number of permutations are desired, then the
number of permutations is supplied to SNPGWA from the command line (See Section
III.A.4, Empirical P-Value Calculation from the Command Line \{FIX
cross-references\}).  If a variable number of permutations are desired, the
number of permutations for each test is calculated by the formula

\begin{equation*}
  NumPerm = \left(18/PVal\right)-9
\end{equation*}

\noindent{}where $PVal$  is the p-value of the unpermuted test.

\subsection{Usage}
\begin{verbatim}

snplash -engine snpgwa -geno <filename> -phen <filename>  \
        -out <filename> -map <filename> [OPTIONS]

where options are

  --condition_number Float. Maximum allowable condition number
                     in logistic and linear regression. Condition
                     number is measured in the 1-norm. Default is 1e12.
  --cov              Comma-separated list of coverariates
  --geno_file        Print extra genotypic data to <outfile>.geno[1,2,3]
  --haplo_thresh     Integer.  Sets the threshold number of chromosomes
                     on which a haplotype must exist to be used for
                     global haplotype association testing.
  --haplo_file       Print exta haplotype data to <outfile>.haplo[1,2,3]
  --hwe_file         Print extra Hardy-Weinberg data to <outfile>.hwe[ctrl,case]
  --snpgwa_nohap     Do not calculate haplotype tests (shortens run-time)
  --val              Print statistics values to <outfile>.statvals

\end{verbatim}

\subsection{Output}

\begin{supertabular}{r p{5in}}
  General & xx ~ \\
    Chr: & The chromosomal location for the feature \\
    Marker: & The marker label \\
    Position: & The position of the feature on the chromosome \\
    Difference: & xx \\ [1em]
  Individuals & xx ~ \\
    Cases  & xx  \\
    Controls  & xx  \\ [1em]
  Reference Allele Frequency & xx \\
    Cases  & xx  \\
    Controls  & xx  \\ [1em]
  Percent Missing & xx  \\
    Combined  & xx  \\
    Cases  & xx  \\
    Controls  & xx  \\
    Pvalue  & xx  \\
    Odds Ratio  & xx  \\ [1em]
    ?Category? & xx  \\
    P  & xx  \\
    Q  & xx  \\
    Ref  & xx  \\ [1em]
  Hardy-Weinberg Analysis & xx ~ \\ [1em]
    Cntl PP  & xx  \\
    Case PP  & xx  \\
    ENumPP  & xx  \\
    Cntl PQ  & xx  \\
    Case PQ  & xx  \\
    ENumPQ  & xx  \\
    Cntl QQ  & xx  \\
    Case QQ  & xx  \\
    ENumQQ  & xx  \\
    X\^2\_PValue  & xx  \\
    Combined Prob\_HWE  & xx  \\
    Case Prob\_HWE  & xx  \\
    Control Prob\_HWE  & xx  \\ [1em]
  Genotypic Association & xx ~ \\
  ~ & xx ~ \\
  2 Deg Fr & xx ~ \\
    PV\_2DF  & xx  \\
  ~ & xx ~ \\
  Dominant Test & xx ~ \\
    PV\_Dom  & xx  \\
    OR  & xx  \\
    LCI  & xx  \\
    UCI  & xx  \\
    Sens  & xx  \\
    Spec  & xx  \\
   C-St  & xx  \\
  ~ & xx ~ \\
  Additive Test & xx ~ \\
    PV\_Add  & xx  \\
    OR  & xx  \\
    LCI  & xx  \\
    UCI  & xx  \\
    NN v RN Sens  & xx  \\
    NN v RN Spec  & xx  \\
    NN v RR Sens  & xx  \\
    NN v RR Spec  & xx  \\
    NR v RR Sens  & xx  \\
    NR v RR Spec  & xx  \\
    NR v RR C-St  & xx  \\
  ~ & xx ~ \\
  Recessive Test & xx ~ \\
    PV\_Rec  & xx  \\
    OR  & xx  \\
    LCI  & xx  \\
    UCI  & xx  \\
    Sens  & xx  \\
    Spec  & xx  \\
    C-St  & xx  \\
  ~ & xx ~ \\
  Lack Fit & xx ~ \\
    PV\_LOF  & xx  \\
  ~ & xx ~ \\
    D\_Prime  & xx  \\
    R\_Squared  & xx  \\
    Allelic Test PValue  & xx  \\
  ~ & xx ~ \\
  Two Marker Haplotype Analysis & xx ~ \\
    LRS\_PValue  & xx  \\
    Cases FreqPP  & xx  \\
    Cases FreqPQ  & xx  \\
    Cases FreqQP  & xx  \\
    Cases FreqQQ  & xx  \\
    Controls FreqPP  & xx  \\
    Controls FreqPQ  & xx  \\
    Controls FreqQP  & xx  \\
    Controls FreqQQ  & xx  \\
  ~ & xx ~ \\
  Three Marker Haplotype Analysis & xx ~ \\
    LRS\_PValue  & xx  \\
    Cases FreqPPP  & xx  \\
    Cases FreqPPQ  & xx  \\
    Cases FreqPQP  & xx  \\
    Cases FreqPQQ  & xx  \\
    Cases FreqQPP  & xx  \\
    Cases FreqQPQ  & xx  \\
    Cases FreqQQP  & xx  \\
    Cases FreqQQQ  & xx  \\
    Controls FreqPPP  & xx  \\
    Controls FreqPPQ  & xx  \\
    Controls FreqPQP  & xx  \\
    Controls FreqPQQ  & xx  \\
    Controls FreqQPP  & xx  \\
    Controls FreqQPQ  & xx  \\
    Controls FreqQQP  & xx  \\
    Controls FreqQQQ  & xx  \\
\end{supertabular} 
